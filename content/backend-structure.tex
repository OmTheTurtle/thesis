\section{A backend felépítése}

Az alkalmazás backendje eltér a hagyományos backend API-tól. A Next.js API routes által a backendünket úgynevezett cloud function-ök alkotják.

Ennek segítségével a \lstinline|pages/api| mappába helyezett fájljaink szolgálnak backendként. Minden ide helyezett fájl
egyben a nevének megfelelő REST API végpont lesz, tehát például a \lstinline|pages/api/books.ts|-ben lévő kódot a \lstinline|pages/api/books| URL-en keresztül tudjuk elérni.

Támogatja továbbá a backend-oldali dinamikus routing-ot, azaz például a \lstinline|pages/api/books/[id].tsx| fájl a \lstinline|pages/api/<id>| URL-nek felel meg, ahol
az \lstinline|id| változót alábbi módon érhetjük el:

\begin{lstlisting}[language=Java, caption=Next.js dinamikus routing]
export default function handler(req, res) {
  const {
    query: { id },
  } = req

  res.end(`Book: ${id}`)
}
\end{lstlisting}

\subsection{next-connect}
Alapesetben a Next.js csak egy egyszerű interface-t biztosít nekünk, amin keresztül elérhetjük a HTTP kérés request és response objektumokat annak kezeléséhez.
Ez azonban nehézkessé teszi a különböző kérések feldolgozását (pl. GET, POST és PUT), valamint különböző kódrészletek egyszerű újrafelhasználását.

Ennek kényelmesebbé tételére döntöttem a next-connect könyvtár használata mellett, amellyel a fenti igények könnyedén megvalósíthatóak.
Az alábbi két kódrészletben szeretném bemutatni a főbb különbségeket.

\begin{lstlisting}[language=Java, caption=Default Next.js API routes]
export default function handler(req: NextApiRequest, res: NextApiResponse) {
  if (req.method === 'GET') {
    res.statusCode = 200
    res.setHeader('Content-Type', 'application/json')
    res.end(JSON.stringify({ name: 'John Doe' }))
  } else if (req.method === 'POST') {
    // Process a POST request
  }
}
\end{lstlisting}

\begin{lstlisting}[language=Java, caption=Kérés kezelése next-connect segítségével]
import nextConnect from "next-connect"

const handler = nextConnect<NextApiRequest, NextApiResponse>()

handler
  .get((req, res) => {
    res.json({ name: 'John Doe' })
  })
  .post((req, res) => {
    // Process POST request
  })

export default handler
\end{lstlisting}

\subsubsection{Middleware támogatás}
A Next.js alapesetben nem rendelkezik beépített middleware támogatással, emiatt bizonyos kódrészletek újrahasználása körülményes lehet.
A next-connect azonban ezt a folyamatot rendkívül egyszerűvé teszi, így könnyen lehet védett útvonalakat létrehozni például csak bejelentkezett
felhasználók számára.

\begin{lstlisting}[language=Java, caption=Middleware kezelés next-connect segítségével]
import nextConnect from "next-connect"
import requireLogin from "middleware/requireLogin"
import requireAdmin from "middleware/requireAdmin"
const handler = nextConnect<NextApiRequest, NextApiResponse>()

handler
  .get((req, res) => {
    res.json({ name: 'John Doe' })
  })
  .use(requreLogin)
  .post((req, res) => {
    // Process POST request
  })
  .use(requireAdmin)
  // Process other requests

export default handler
\end{lstlisting}

A fenti kódrészletben a POST kérés csak bejelentkezett felhasználók számára elérhető. Ezek egymás után is fűzhetőek, így komplex
igények is rendkívül egyszerűen megvalósíthatóak.


\subsection{Kommunikáció az adatbázissal}

A backend az adatbázisból a Prisma segítségével szolgáltatja az adatokat. Ahhoz hogy ezt meg tudjuk tenni, első körben szükséges az
adatbázisséma és a Prisma séma közötti kapcsolatot felépíteni.

A séma adatbázisba történő átvezetésére két lehetőségünk van. Az egyik, hogy a dbdiagram oldalról lehetőségünk van .sql kiterjesztésű fájlt letölteni,
ezt a létrehozott adatbázisunkon futtatni, majd a Prisma introspect funkcióját használva legenerálni hozzá a Prisma schema fájlt a backendünk számára.
A másik, egyszerűbb megoldás a Prisma migrate\footnote{A dolgozat írása idejében ez a funkció  még experimental státuszban volt, de a használata során nem ütköztem problémákba.} használata.
Ez esetben nekünk manuálisan kell létrehozni a Prisma schema fájlt a korábbi diagram alapján, majd a
\begin{lstlisting}
prisma migrate save --experimental
prisma migrate up --experimental
\end{lstlisting}

parancsokat kiadva létrehozzuk és futtatjuk a Prisma migrációt.
Ez utóbbi megoldás előnyei, hogy egy központi helyen tudjuk kezelni a séma változásait, valamint ezt adatbázis-agnosztikus módon
tehetjük meg.

\begin{lstlisting}[caption=Séma leírása a Prisma DSL-ben]
generator client {
  provider = "prisma-client-js"
}

datasource db {
  provider = "postgresql"
  url      = env("DATABASE_URL")
}

model Book {
  id          Int           @id @default(autoincrement())
  title       String
  author      String?
  isbn        String?
  publisher   String?
  publishedAt Int?
  stockCount  Int?          @default(1)
  count       Int?          @default(1)
  notes       String?
  image       String?
  createdAt   DateTime?     @default(now())
  updatedAt   DateTime?     @default(now())
  orders      BookToOrder[]
  categories  Category[]
}

model BookToOrder {
  id       Int   @id @default(autoincrement())
  orderId  Int
  bookId   Int
  quantity Int?  @default(1)
  books    Book  @relation(fields: [bookId], references: [id])
  orders   Order @relation(fields: [orderId], references: [id])

  @@unique([bookId, orderId], name: "BookToOrder_book_order_unique")
}

model Category {
  id    Int    @id @default(autoincrement())
  name  String
  books Book[]
}

model Comment {
  id        Int       @id @default(autoincrement())
  text      String?
  createdAt DateTime? @default(now())
  userId    Int
  orderId   Int
  order     Order     @relation(fields: [orderId], references: [id])
  user      User      @relation(fields: [userId], references: [id])
}

model Order {
  id         Int           @id @default(autoincrement())
  userId     Int
  returnDate DateTime?
  status     orderstatus?  @default(PENDING)
  createdAt  DateTime?     @default(now())
  updatedAt  DateTime?     @default(now())
  user       User          @relation(fields: [userId], references: [id])
  books      BookToOrder[]
  comments   Comment[]
}

model User {
  id        Int       @id @default(autoincrement())
  email     String    @unique
  password  String
  name      String?
  createdAt DateTime? @default(now())
  role      userrole? @default(BASIC)
  comments  Comment[]
  orders    Order[]
}

enum orderstatus {
  PENDING
  RENTED
  RETURNED
  LATE
}

enum userrole {
  BASIC
  ADMIN
  EDITOR
}
\end{lstlisting}

A Prisma DSL-ben a dbdiagram.io-hoz hasonló módon tudjuk felvenni a relációkat. Ennek nagy előnye, hogy a táblák közötti kapcsolatokat
egyszerűen tudjuk modellezni, amit a \lstinline|prisma migrate| át tud vezetni az adatbázisunkba.

Ahogy a fenti kódrészletben is látszik, a Category és a Book modellek közötti kapcsolatot biztosító kapcsolótábla nem jelenik
meg a Prisma schema fájlban. Ez egy úgynevezett implicit kapcsolat, melyet a Prisma motorja automatikusan kezel és a migráció során
létrehozza a megfelelő táblát a két entitás között.

\subsection{Kapcsolódás az adatbázishoz}
A Prisma esetében az adatbázis-kapcsolatot egy connection string segítségével tudjuk megadni, amit alapesetben
a megadott környezeti változóból (\lstinline|DATABASE_URL|) olvas ki. Ennek nagy előnye, hogy könnyen tudunk lokális
és remote adatbázisok között váltani, illetve egyszerűvé teszi a production-ben lévő kód és az adatbázis közötti
kapcsolat létrehozását.

\subsection{Az adatok elérése}

Miután minden szükséges konfigurációt elvégeztünk, el tudjuk érni az adatainkat a Prisma Client-en keresztül.

Ahhoz hogy az adatbázis-kapcsolatok számát minimalizáljuk, készítettem egy központi fájlt, ami a Prisma Clien-et exportálja.
Ennek köszönhetően csak egy példány fog élni az alkalmazáson belül, optimalizálva ezzel annak működését.

\begin{lstlisting}[caption=Prisma Client export]
import { PrismaClient } from "@prisma/client"

const prisma = new PrismaClient()

export default prisma
\end{lstlisting}

Ezután az adatbázis eléréséhez elegendő ezt az egy fájlt importálni a serverless függvényünkben.

\begin{lstlisting}[caption=Adatelérés next-connect middleware és Prisma segítségével]
const handler = nextConnect<NextApiRequest, NextApiResponse>()

handler
  .use(auth())
  .use(requireLogin())
  .use(requireRole(userrole.ADMIN))
  .get(async (req, res) => {
    try {
      const categories = await db.category.findMany()
      res.json(categories)
    } catch (error) {
      console.error(error)
      res.status(500).send(error.message)
    }
  })
\end{lstlisting}
