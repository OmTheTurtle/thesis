\section{A backend felépítése}

\subsection{Next.js API routes}

A Next.js keretrendszer a 9-es verzió óta lehetővé teszi szerveroldali kód írását az alkalmazásunkhoz.
Ennek segítségével a \lstinline|pages/api| mappába helyetett fájljaink szolgálnak backendként. Minden ide helyezett fájl
egyben a nevének megfelelő API végpont lesz, tehát például a \lstinline|pages/api/books.ts|-ben lévő kódot a pages/api/books URL-en keresztül tudjuk elérni.

Támogatja továbbá a backend-oldali dinamikus routing-ot, azaz például a \lstinline|pages/api/books/[id].tsx| fájl a \lstinline|pages/api/<id>| URL-nek felel meg, ahol
az \lstinline|id| változót alábbi módon érhetjük el:

\begin{lstlisting}[language=Java, caption=Next.js dinamikus routing]
export default function handler(req, res) {
  const {
    query: { id },
  } = req

  res.end(`Book: ${id}`)
}
\end{lstlisting}

\subsection{next-connect}
Alapesetben a Next.js csak egy egyszerű interface-t biztosít nekünk, amin keresztül elérhetjük a HTTP kérés request és response objektumokat annak kezeléséhez.
Ez azonban nezhézkessé teszi a különböző kérések feldolgozását (pl. GET, POST és PUT), valamint különbőző kódrészletek egyszerű újrafelhasználását.

Ennek kényelmesebbé tételére döntöttem a next-connect könyvtár használata mellett, amellyel a fenti igények könnyedén megvalósíthatóak.
Az alábbi két kódrészletben szeretném bemutatni a főbb különbségeket.

\begin{lstlisting}[language=Java, caption=Default Next.js API routes]
export default function handler(req: NextApiRequest, res: NextApiResponse) {
  if (req.method === 'GET') {
    res.statusCode = 200
    res.setHeader('Content-Type', 'application/json')
    res.end(JSON.stringify({ name: 'John Doe' }))
  } else if (req.method === 'POST') {
    // Process a POST request
  }
}
\end{lstlisting}

\begin{lstlisting}[language=Java, caption=Kérés kezelése next-connect segítségével]
import nextConnect from "next-connect"

const handler = nextConnect<NextApiRequest, NextApiResponse>()

handler
  .get((req, res) => {
    res.json({ name: 'John Doe' })
  })
  .post((req, res) => {
    // Process POST request
  })

export default handler
\end{lstlisting}

\subsubsection{Middleware támogatás}
A Next.js alapesetben nem rendelkezik beépített middleware támogatással, emiatt bizonyos kódrészletek újrahasználása körülményes lehet.
A next-connect azonban ezt a folyamatot rendkívül egyszerűvé teszi, így könnyen lehet védett útvonalakat létrehozni például csak bejelentkezett
felhasználók számára.

\begin{lstlisting}[language=Java, caption=Middleware kezelés next-connect segítségével]
import nextConnect from "next-connect"
import requireLogin from "middleware/requireLogin"
import requireAdmin from "middleware/requireAdmin"
const handler = nextConnect<NextApiRequest, NextApiResponse>()

handler
  .get((req, res) => {
    res.json({ name: 'John Doe' })
  })
  .use(requreLogin)
  .post((req, res) => {
    // Process POST request
  })
  .use(requireAdmin)
  // Process other requests

export default handler
\end{lstlisting}

A fenti kódrészletben a POST kérés csak bejelentkezett felhasználók számára elérhető. Ezek egymás után is fűzhetőek, így komplex
igények is rendkívül egyszerűen megvalósíthatóak.
