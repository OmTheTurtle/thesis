\pagenumbering{roman}
\setcounter{page}{1}

\selecthungarian

%----------------------------------------------------------------------------
% Abstract in Hungarian
%----------------------------------------------------------------------------
\chapter*{Kivonat}\addcontentsline{toc}{chapter}{Kivonat}

Manapság ha egy többfelhasználós, dinamikus tartalmat megjelenítő szolgáltatásra van szükségünk, önkéntelenül is a webes technológiák
felé fordulunk. Ez nem véletlen, ugyanis a rendkívül gyorsan fejlődő webes technológiák lehetőséget nyújtanak
arra, hogy kényelmes, gyors, megbízható és jó felhasználói élményt nyújtó megoldásokat készítsünk általuk.

Szakdolgozatomban a Simonyi Károly Szakkollégiumnak elkészített könyvkölcsönző alkalmazáson keresztül szeretném bemutatni
ezeknek a technológiáknak egy szeletét és betekintést nyújtani a fejlesztés menetébe.

A szakkollégiumhoz tartozó könyvtár régóta nyitva áll a hallgatók előtt hasznos illetve szórakoztató irodalmat kínálva.
Ennek elérése illetve karbantartása azonban ez idáig nehézkes és bonyolult volt, azonban szerintem nagy potenciál rejlik benne,
ha ezt mind a hallgatók, mind az üzemeltetők egy kényelmesen használható felületek keresztül érhetik el.

A dolgozat ennek a szoftvernek a létrejöttét mutatja be. Ebbe beletartozik a megfelelő technológia kiválasztása -- külön kitérve a
Next.js által szolgáltatott funkciókra --, valamint az igények pontosítása a specifikáció által.
Ezután a szoftver felépítéséről valamint a megvalósított funkciók működéséről lesz szó.

Végül az elkészült alkalmazás tesztelése, a fejlesztés segítő eszközök bemutatása, valamint a potenciális továbbfejlesztési
lehetőségek kerülnek bemutatásra.


\vfill
\selectenglish


%----------------------------------------------------------------------------
% Abstract in English
%----------------------------------------------------------------------------
\chapter*{Abstract}\addcontentsline{toc}{chapter}{Abstract}

Nowadays if someone wanted to make a multi-user solution that can provide dynamic data for the users, can't look away from the web.
This is because today's web technologies allow the developers to create easy-to-use, fast and reliable applications that provide a great user expreience.

In my thesis I'd like to demonstrate a portion of these technologies and the development process through a web-based book renting application
created for Simonyi Károly College for Advanced Studies.

The library of Simonyi Károly College for Advanced Studies is open to students for renting books that can help them in their
studies or provide other kinds of literature. In its current form the handling of the library proved to be difficult and complicated.

The goal of my thesis is to provide an easy-to-use and accessible web-based application to ease the access to the library for end-users
and make managing the books and orders simpler and easier for college members. With this system using the library will be more straightforward
and more students could be potentially reached about this service.

The goal of this thesis is to guide the Reader through how this application was made from scratch. This includes choosing the
most appropriate frameworks and libraries -- with extra details about Next.js and it's provided services -- and defining the
exact demands through specification.

Next the dissertation will show the application's structure and how the functionalities defined in the specification work.

Lastly the thesis presents how the app was tested, what other technologies were used to make the development phase easier and how
the application can be further optimized and developed.

\vfill
\selectthesislanguage

\newcounter{romanPage}
\setcounter{romanPage}{\value{page}}
\stepcounter{romanPage}