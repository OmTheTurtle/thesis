%----------------------------------------------------------------------------
\appendix
%----------------------------------------------------------------------------
\chapter*{\fuggelek}\addcontentsline{toc}{chapter}{\fuggelek}
\setcounter{chapter}{\appendixnumber}
%\setcounter{equation}{0} % a fofejezet-szamlalo az angol ABC 6. betuje (F) lesz
\numberwithin{equation}{section}
\numberwithin{figure}{section}
\numberwithin{lstlisting}{section}
%\numberwithin{tabular}{section}

\section{Telepítési útmutató}

Az alkalmazás lokális futtatásához az alábbiak szükségesek:
\begin{itemize}
  \item Node.js 12
  \item yarn
  \item PostgreSQL
  \item Amazon S3 bucket létrehozása
\end{itemize}

PostgreSQL-ben hozzunk létre egy új adatbázist, majd a hozzá tartozó connection string-et a \lstinline|prisma/.env| fájlba mint
környezeti változó vegyük fel.

Szükséges továbbá az S3-hoz tartozó környezeti változók beállítása a \lstinline|.env| fájlban, ezt az alkalmazás gyökérmappájában helyezzük el.
Ezután a \lstinline|yarn install| parancs kiadásával tudjuk a kellő függőségeket letölteni, majd a \lstinline|yarn dev| utasítással
tudjuk az alkalmazást futtatni.

\section{Az alkalmazás elérhetősége}

A program forráskódja a \href{https://OmTheTurtle/simonyi-konyvtar}{GitHub}-on megtalálható.
A legfrisseb build pedig megtekinthető a \href{https://simonyi-konyvtar.vercel.app}{https://simonyi-konyvtar.vercel.app} oldalon.
