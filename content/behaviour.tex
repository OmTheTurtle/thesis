\chapter{Az alkalmazás működése}

\section{Authentikáció}

A felhasználók magukat egy email-jelszó párossal tudják azonosítani, melyet regisztrációkor adhatnak meg.

A regisztráció kezelésénék két dologra kellett figyelmet fordítanom: egy email cím csak egyszer szerepelhessen
az adatbázisban, valamint a jelszót megfelelő módon tároljuk az adatbázisban.

Az első kritériumra megoldásként szolgál az adatbázisban az email mező egyedivé tétele, valamint a regisztráció során
a megadott emailcím ellenőrzése.

A jelszó biztonságos tárolása érdekében azt regisztráció során hash-elve mentem el az adatbázisba, erre az \lstinline|argon2| könyvtárat
használtam fel.

\subsection{Megvalósítás}

A session kezeléshez a \lstinline|passport| könyvtárat és cookie alapú megoldást használtam. Ennek során a felhasználót a böngészőben
tárolt cookie információ azonosítja, amit backenden ellenőrzi tudunk.

A frontenden történő ellenőrzéshez egy hook-ot készítettem, mellyel ellenőrzhető, hogy a felhasználó be van-e jelentkezve.

\begin{lstlisting}[caption=Authentikáció hook]
export function useUser() {
  const { data, mutate } = useSWR<{ user: User }>("/api/user", fetcher)
  const loading = !data
  const user = data?.user
  return [user, { mutate, loading }] as const
}
\end{lstlisting}

\section{Authorizáció}

Az alkalmazás megfelelő használata érdekében szükséges volt bizonyos funkciók elérésének szűkítésére. Ehhez a role-based access control
megoldást választottam, mely alapján a felhasználókat különböző kategórákba tudjuk besorolni, majd ezeknek a kategóriáknak adunk jogosultságokat.

Ennek megfelelően három jogosultság-kategóriát hoztam létre: \lstinline|BASIC|, \lstinline|ADMIN| és \lstinline|EDITOR|.

A \lstinline|BASIC| felhasználók képesek foglalást leadni és a sajátjaikhoz megjegyzést fűzni, valamint a hozzájuk tartozó foglalásokat listázni.

Az \lstinline|EDITOR| jogosultsággal rendelkezők ezen felül képesek az egyes foglalások állapotát állítani, valamint bármyely foglaláshoz
megjegyzést fűzni.

Az \lstinline|ADMIN| joggal rendelkezők a fentieken kívül képesek könyvek és kategóriák hozzáadására, törlésére és szerkesztésére.

\subsection{Megvalósítás}

A kódbázison belül a frontenden és backendes is szükséges elleőrizni a megfelelő jogosultságokat.

A backenden ehhez létrehoztam egy middleware-t, amely a már bejelentkezett emberek jogosultságát ellenőrzi.
\begin{lstlisting}[caption=Authorizáció middleware]
const requireRole = (...roles: userrole[]) => {
  return (req: NextApiRequest, res: NextApiResponse, next: NextHandler) => {
    if (roles.some(it => it === req.user.role)) {
      return next()
    } else {
      return res.status(401).json({ message: "Nincs megfelelő jogosultságod" })
    }
  }
}
\end{lstlisting}

A frontenden történő validáció esetén két forgatókönyv lehetséges: egy teljes oldal vagy az oldalon belül bizonyos komponensek
elrejtése a felhasználó elől.

Az előbbi kezelésére létrehoztam egy React hook-ot, amivel az oldalak megjelenítését tudjuk kontrollálni.

\begin{lstlisting}[caption=Authorizációs hook és használata]
// src/lib/hooks.tsx
export function useRequireRoles(roles: userrole[] = []) {
  const [user] = useUser()

  return roles.some((it) => it === user?.role)
}

// src/pages/admin/index.tsx
const hasAccess = useRequireRoles([userrole.ADMIN, userrole.EDITOR])
if (!hasAccess) {
  return <ErrorPage statusCode={401} message="Nincs megfelelő jogosultságod!" />
}
\end{lstlisting}

Az oldalon belüli bizonyos tartalmak elrejtésére készítettem egy komponenst, ami a tartalmát a felhasználó jogosultsági szintjének
megfelelően jeleníti csak meg.

\begin{lstlisting}[caption=Authorizáció komponens és használata]
// src/components/HasRole.tsx
export default function HasRole({ roles, children }: Props) {
  const [user] = useUser()
  const hasRole = roles.some((it) => it === user?.role)

  return <>{hasRole && children}</>
}

// src/components/Navbar.tsx
<HasRole roles={[userrole.ADMIN, userrole.EDITOR]}>
  <NextLink href="/admin">
    <Link>Admin</Link>
  </NextLink>
</HasRole>
\end{lstlisting}

A fenti módszerek alkalmazásásval egy robosztus jogosultság-kezelő megoldást sikerült implementálnom a szoftverbe.


\section{Könyvek listázása}

\section{Keresés a könyvek között}

\section{Foglalási folyamat}

\section{Admin funkciók}

\subsection{Fájlfeltöltés}

\section{Dark Mode}


