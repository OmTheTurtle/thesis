\chapter{Az alkalmazás működése}

\section{Authentikáció}

A felhasználók magukat egy email-jelszó párossal tudják azonosítani, melyet regisztrációkor adhatnak meg.

A regisztráció kezelésénék két dologra kellett figyelmet fordítanom: egy email cím csak egyszer szerepelhessen
az adatbázisban, valamint a jelszót megfelelő módon tároljuk az adatbázisban.

Az első kritériumra megoldásként szolgál az adatbázisban az email mező egyedivé tétele, valamint a regisztráció során
a megadott emailcím ellenőrzése.

A jelszó biztonságos tárolása érdekében azt regisztráció során hash-elve mentem el az adatbázisba, erre az \lstinline|argon2| könyvtárat
használtam fel.

\subsection{Megvalósítás}

A session kezeléshez a \lstinline|passport| könyvtárat és cookie alapú megoldást használtam. Ennek során a felhasználót a böngészőben
tárolt cookie információ azonosítja, amit backenden ellenőrzi tudunk.

A frontenden történő ellenőrzéshez egy hook-ot készítettem, mellyel ellenőrzhető, hogy a felhasználó be van-e jelentkezve.

\begin{lstlisting}[caption=Authentikáció hook]
export function useUser() {
  const { data, mutate } = useSWR<{ user: User }>("/api/user", fetcher)
  const loading = !data
  const user = data?.user
  return [user, { mutate, loading }] as const
}
\end{lstlisting}

\section{Authorizáció}

Az alkalmazás megfelelő használata érdekében szükséges volt bizonyos funkciók elérésének szűkítésére. Ehhez a role-based access control
megoldást választottam, mely alapján a felhasználókat különböző kategórákba tudjuk besorolni, majd ezeknek a kategóriáknak adunk jogosultságokat.

Ennek megfelelően három jogosultság-kategóriát hoztam létre: \lstinline|BASIC|, \lstinline|ADMIN| és \lstinline|EDITOR|.

A \lstinline|BASIC| felhasználók képesek foglalást leadni és a sajátjaikhoz megjegyzést fűzni, valamint a hozzájuk tartozó foglalásokat listázni.

Az \lstinline|EDITOR| jogosultsággal rendelkezők ezen felül képesek az egyes foglalások állapotát állítani, valamint bármyely foglaláshoz
megjegyzést fűzni.

Az \lstinline|ADMIN| joggal rendelkezők a fentieken kívül képesek könyvek és kategóriák hozzáadására, törlésére és szerkesztésére.

\subsection{Megvalósítás}

A kódbázison belül a frontenden és backendes is szükséges elleőrizni a megfelelő jogosultságokat.

A backenden ehhez létrehoztam egy middleware-t, amely a már bejelentkezett emberek jogosultságát ellenőrzi.
\begin{lstlisting}[caption=Authorizáció middleware]
const requireRole = (...roles: userrole[]) => {
  return (req: NextApiRequest, res: NextApiResponse, next: NextHandler) => {
    if (roles.some(it => it === req.user.role)) {
      return next()
    } else {
      return res.status(401).json({ message: "Nincs megfelelő jogosultságod" })
    }
  }
}
\end{lstlisting}

A frontenden történő validáció esetén két forgatókönyv lehetséges: egy teljes oldal vagy az oldalon belül bizonyos komponensek
elrejtése a felhasználó elől.

Az előbbi kezelésére létrehoztam egy React hook-ot, amivel az oldalak megjelenítését tudjuk kontrollálni.

\begin{lstlisting}[caption=Authorizációs hook és használata]
// src/lib/hooks.tsx
export function useRequireRoles(roles: userrole[] = []) {
  const [user] = useUser()

  return roles.some((it) => it === user?.role)
}

// src/pages/admin/index.tsx
const hasAccess = useRequireRoles([userrole.ADMIN, userrole.EDITOR])
if (!hasAccess) {
  return <ErrorPage statusCode={401} message="Nincs megfelelő jogosultságod!" />
}
\end{lstlisting}

Az oldalon belüli bizonyos tartalmak elrejtésére készítettem egy komponenst, ami a tartalmát a felhasználó jogosultsági szintjének
megfelelően jeleníti csak meg.

\begin{lstlisting}[caption=Authorizáció komponens és használata]
// src/components/HasRole.tsx
export default function HasRole({ roles, children }: Props) {
  const [user] = useUser()
  const hasRole = roles.some((it) => it === user?.role)

  return <>{hasRole && children}</>
}

// src/components/Navbar.tsx
<HasRole roles={[userrole.ADMIN, userrole.EDITOR]}>
  <NextLink href="/admin">
    <Link>Admin</Link>
  </NextLink>
</HasRole>
\end{lstlisting}

A fenti módszerek alkalmazásásval egy robosztus jogosultság-kezelő megoldást sikerült implementálnom a szoftverbe.

\section{Könyvek listázása}

A kezdőoldalon az elérhető könyveket tudjuk listázni egy rácsszerkezetben.
Itt látható a könyvöz kapcsolt borítókép, a könyv címe, szerzője és kategóriái.

\begin{figure}[!ht]
  \centering
  \includegraphics[width=150mm, keepaspectratio]{figures/index.png}
  \caption{Az alkalmazás kezdőlapja}
  \label{fig:IndexPage}
\end{figure}

\subsection{Részletes nézet}

A főoldalon egy könyvre kattintva megkapjuk annak részletes nézetét. Itt láthatjuk az összes hozzá tartozó információt, valamint
lehetőségünk van a könyv kosárba helyezésére (feltéve, hogy van belőle elérhető példány).

\begin{figure}[!ht]
  \centering
  \includegraphics[width=100mm, keepaspectratio]{figures/book-detail-view.png}
  \caption{Könyv részletes nézete}
  \label{fig:BookDetailView}
\end{figure}

\subsection{Keresés a könyvek között}

A főoldalon tudunk a meglévő könyvek közötti kulcsszavas keresésre. Ezt a PostgreSQL Full Text Search funkciójával implementáltam.

Ehhez szükséges volt létrehozni az indexeléshez szükséges oszlopot a \lstinline|Book| táblán, ami alapján az adatbázismotor a kersést
el tudja végezni.

\begin{lstlisting}[caption=A kereséshez szükséges SQL utasítások]
ALTER TABLE "Book"
ADD COLUMN document tsvector;
update "Book"
set document = to_tsvector(title || ' ' || author || ' ' || publisher || '' || notes);

ALTER TABLE "Book"
ADD COLUMN document_with_idx tsvector;
update "Book"
set document_with_idx = to_tsvector(title || ' ' || coalesce(author, '') || ' ' || coalesce(publisher, '') || '' || coalesce(notes, ''));
CREATE INDEX document_idx
ON "Book"
USING GIN(document_with_idx);

ALTER TABLE "Book"
  ADD COLUMN document_with_weights tsvector;
update "Book"
set document_with_weights = setweight(to_tsvector(title), 'A') ||
  setweight(to_tsvector(coalesce(author, '')), 'B') ||
  setweight(to_tsvector(coalesce(publisher, '')), 'C') ||
  setweight(to_tsvector(coalesce(notes, '')), 'D');
CREATE INDEX document_weights_idx
  ON "Book"
  USING GIN (document_with_weights);

CREATE FUNCTION book_tsvector_trigger() RETURNS trigger AS $$
begin
  new.document :=
  setweight(to_tsvector('english', coalesce(new.title, '')), 'A')
  || setweight(to_tsvector('english', coalesce(new.author, '')), 'B')
  || setweight(to_tsvector('english', coalesce(new.publisher, '')), 'C')
  || setweight(to_tsvector('english', coalesce(new.notes, '')), 'D');
  return new;
end
$$ LANGUAGE plpgsql;

CREATE TRIGGER tsvectorupdate BEFORE INSERT OR UPDATE
    ON "Book" FOR EACH ROW EXECUTE PROCEDURE book_tsvector_trigger();

\end{lstlisting}

A keresés implementálásához szükséges volt még egy egyedi lekérdezés írása, ugyanis a Prisma jelenleg nem támogatja a \lstinline|tsvector|
alapú keresést. Ehhez az alábbi megoldást használtam a backenden.

\begin{lstlisting}[caption=Könyvek közti keresés megvalósítása]
const sql = escape(`
select id, title, author, "stockCount", "updatedAt", image
from "Book"
where document_with_idx @@ plainto_tsquery('%s')
order by ts_rank(document_with_idx, plainto_tsquery('%s')) desc;`, term)
const books = await db.$queryRaw(sql)
\end{lstlisting}

Alapesetben ha elkezdünk a keresőmezőbe gépelni, a frontend minden egyes leütés után kérést intéz a backend felé.
Ez azonban felesleges forgalmat és adatbáziselérést okoz, ezért az input késleltetésére a \lstinline|use-debounce| pagckage-et használtam.

\begin{lstlisting}[caption=A keresést megvalósító kódrészlet a frontenden]
const [term, setTerm] = useState("")
const { data, error } = useSWR<BookWithCategories[]>(`/api/books?q=${term}`, fetcher)
const debounced = useDebouncedCallback((value) => setTerm(value), 500)

return (
  <Input
    placeholder="Keress a könyvek között!"
    mt="1rem"
    onChange={(e) => debounced.callback(e.target.value)}
  />
  {/* ... */}
)
\end{lstlisting}

\section{Foglalási folyamat}

Az alkalmazás legfontosabb eleme a könyvek foglalásának nyomonkövetése. Az alábbi ábrán ennek a folyamatát foglaltam össze.

% TODO: folyamatábra a foglalásról: kosárba helyezés -> foglalás leadása -> megbeszélés -> leadás

\subsection{Kosár}

A könyv részletes nézetében lehetőség van a kosárba helyezésre. Ennek a tartalmát a felső navigációs sávon lévő bevásárlókosár
ikonra kattintva lehet megtekinteni.

\begin{figure}[!ht]
  \centering
  \includegraphics[width=150mm, keepaspectratio]{figures/cart.png}
  \caption{Kosár oldal}
  \label{fig:CartPage}
\end{figure}

Itt lehetőség van az egyes könyvek darabszámának állítására, illetve a kosár tartalmának módosítására.

Ezután a felhasználó ki tudja választani, hogy meddig szeretné kiválasztani a könyveket, majd a ``Foglalás leadása'' gombra kattintva
véglegesítheti azt.

\subsection{Foglalás kezelése}

A foglalás leadás után a ``Kölcsönzéseim'' linkre kattintva tudjuk listázni azokat. Itt egy adott kölcsönzésre kattintva
léphetünk a részletes nézetre, ahol megjegyzéseket tudunk fűzni hozzá. Ez a funkció szolgál az átvételi időpont egyeztetésére,
problémák és egyéb igények megbeszélésére.

\begin{figure}[!ht]
  \centering
  \includegraphics[width=150mm, keepaspectratio]{figures/orders-list.png}
  \caption{Kölcsönzések listázása}
  \label{fig:S3Console}
\end{figure}


\begin{figure}[!ht]
  \centering
  \includegraphics[width=150mm, keepaspectratio]{figures/order-detail.png}
  \caption{Kölcsönzés részletei kommentekkel}
  \label{fig:S3Console}
\end{figure}


\section{Admin funkciók}

Lorem ipsum valkjsdf
asfd
asdfa
sfd
a
fd

\subsection{Könyvek kezelése}
a
sfdas
dfa
s
as
dfaasdf



asdf
as
df


\subsection{Fájlfeltöltés}

A könyvekhez opcionálisan megadható egy borítókép, amit a felhasználó a saját gépéről választhat ki.

Ennek a tárolására az Amazon S3 szolgáltatását választottam. A csatolt képet a frontend először elküldi a backendnek,
majd az az \lstinline|aws-sdk| könyvtárat használva feltölti a képet az S3 bucket-be, az adatbázisba csak a képet
azonosító generált kerül be. Így lehetőségünk van a képek egyszerű és adatbázisfüggetlen kezelésére.

\begin{figure}[!ht]
  \centering
  \includegraphics[width=150mm, keepaspectratio]{figures/s3-dashboard.png}
  \caption{Amazon S3 konzol}
  \label{fig:S3Console}
\end{figure}

\subsection{Kategóriák kezelése}

\subsection{Foglalások kezelése}

\section{Adaptív UI}

A felület fejelsztése közben fontos volt, hogy minden oldal megfelelően működjön mobil képernyőkön is.
A Chakra UI szerencsére első kézből támogatja ezt a Responsive Styles funkciójának köszönhetően.

Ennek segítségével egy Chakra komponens tulajdonságait egy tömbben tudjuk megadni, ahol minden elem egy adott képernyőmérethez
fog társulni.

\begin{lstlisting}[caption=Chakra UI Responsive Styles használata]
<Flex direction={["column", null, "row"]}>
  <Box mr={4}>
    <NextImage
      src={
        book.image
          ? `${process.env.NEXT_PUBLIC_S3_URL}/${book.image}`
          : "https://via.placeholer.com/200x300"
      }
      width={300}
      height={450}
    />
  </Box>
  {/* Other content */}
</Flex>
\end{lstlisting}

\begin{figure}[!ht]
  \centering
  \includegraphics[width=75mm, keepaspectratio]{figures/cart-mobile.png}
  \caption{Kosár oldal mobil képernyőn, nyitott menüvel}
  \label{fig:CartMobile}
\end{figure}

\section{Dark Mode}

A Chakra UI egyik rendkívül hasznos tulajdonsága, hogy elsőrendű dark mode támogatással rendelkezik, és valamennyi komponensnek
létezik sötét módú variánsa.
Ennek segítségével gombnyomásra válthatunk a világos és sötét módok között.

\begin{figure}[!ht]
  \centering
  \includegraphics[width=150mm, keepaspectratio]{figures/dark-mode.png}
  \caption{Sötét téma}
  \label{fig:DarkMode}
\end{figure}

\section{Validáció}

A felhasználó által szolgáltatott adatok minden esetben potenciális veszélyforrást jelenthetnek, legyen szó XSS támadásról
vagy az adatbázis struktúrájának integritásáról.

Ezen adatok ellenőrzésére a \lstinline|yup| könyvtárat használtam. Segítségével definiálhatunk egy sémát, ami ellen
a felhasználó által megadott adatot validálhatjuk. Ezáltal a potenciális inkonzisztenciákat még az adatbázisba írás előtt
kiszűrhetjük és tudathatjuk a felhasználóval.

\begin{lstlisting}[caption=yup validációs séma a könyvekre]
export const BookSchema = yup.object().shape({
  title: yup.string().required(),
  author: yup.string(),
  count: yup.number(),
  stockCount: yup.number(),
  isbn: yup.string(),
  publisher: yup.string(),
  publishedAt: yup.number(),
  notes: yup.string(),
  image: yup.string(),
})
\end{lstlisting}
