\chapter{Használt technológiák kiválasztása}

Az alábbi fejezetben az alkalmazás egyes részeihez elérhető és végső soron kiválasztott technológiákat mutatom be.

A technológiák kiválasztásakor fontos szempont volt a Developer Experience (DX) \cite{DX}, azaz hogy a használt eszköz minél kevésbé legyen megterhelő a használója számára.

Ez takarhatja például a kiterjedt és jó minőségű dokumentációt, a megfelelő IDE integrációt, vagy hogy nem kell órákat tölteni az egyes funkciók konfigurálásával.

Fontos szempont volt ezen kívül az adott technológia időtállósága. Mivel a tervek szerint az alkalmazás a jövőben aktívan használva és fejlesztve
lesz akár több fejlesztő által kulcsfontosságú volt a követhető és karbantartható kódbázis és az alkalmazás életciklusának nyomonkövethetősége.

\section{Verziókezelés}

A fejlesztés során fontos a változások átlátható, könnyű követése, a szoftver különböző verzióinak követése.

Erre a mára már kvázi ipari sztenderdnek számító \lstinline|git| nevű szoftvert vettem igénybe. A forráskód felhőben
történő tárolására a GitHub szolgáltatását használtam.

%----------------------------------------------------------------------------
\section{Frontend}
%----------------------------------------------------------------------------
\subsection{JavaScript keretrendszer}
A frontendhez használt technológia kiválasztásánál két fő szempontot tudunk megkülönböztetni.
Az egyik az egyes backend rendszerek által támogatott, szerveroldalon renderlet, template engine-t használó megoldás, a másik a külön frontend framework használata.
Utóbbi jóval nagyobb szabadságot és funkcionalitást ad a fejlesztő kezébe, és lehetőséget ad a legújabb technikák és könyvtárak használatára.

Emiatt úgy döntöttem, hogy az alkalmazás ezen részét a négy legnépszerűbb keretrendszer (React, Angular, Vue és Svelte) egyikével fogom megvalósítani.
Ezek a keretrendszerek alapvető filozófiában és funkcionalitásban lényegében megegyeznek, azonban az elérhető könyvtárak mennyisége és minősége a React esetében a legnagyobb, emiatt végső soron erre esett a választásom. \cite{React}

\subsection{CSS keretrendszer}
A CSS keretrendszer kiválasztása során két fő irányvonal jött szóba: az ún. atomic CSS, illetve az előre már elkészített komponenseket kínáló könyvtárak.
Míg előbbi lehetővé teszi a teljesen egyedi design írását, utóbbi jóval nagyobb fokú kényelmet és a felület gyorsabb elkészítését teszi lehetővé.

Mivel jelen alkalmazás esetében nem volt szempont egy egyedi design elkészítése, ezért az utóbbi megoldás használata mellett döntöttem.
A kiválasztott framework végül a Chakra UI lett, mely nagy mennyiségű kész, egyszerűen bővíthető és magas fokú accessibility-t kínáló komponensekkel rendelkezik. \cite{ChakraUI}

%----------------------------------------------------------------------------
\section{Backend}
%----------------------------------------------------------------------------
A backend keretrendszer kiválasztása során a legfontosabb szempont az volt, hogy minél jobban képes legyen integrálódni a frontend keretrendszerhez,
emiatt mindenképpen szerettem volna elkerülni a külön repository használatát.
Ennek fő okai, hogy mind a fejlesztés, mind a deployment jelentősen egyszerűsödik, valamint közös könyvtár esetén a közös kódrészletek megosztása is triviálissá válik.

A fentiek miatt a lehetőségeim a NodeJS alapú megoldásokra szűkítettem le. Egy népszerű, és általam már kipróbált megoldás az Express keretrendszer, azonban a React világában létezik egy, az igényeimnek méginkább megfelelő framework: a NextJS.
Ez egy React-ra épülő, SSR-t\footnote{Server-Side Rendering} támogató keretrendszer, amely rendelkezik egy úgynevezett API routes nevű funkcióval.

Ennek segítségével a React alkalmazásunkon belül készíthető egy API réteg, amely a fordítás után szerveroldalon futó függvényekké lesz átalakítva.
Ezzel lényegében megspóroljuk, hogy külön API-t és hozzávaló elérési, illetve deployment környezetet kelljen létrehozni, miközben egy Express-hez hasonló interfészt tudunk használni.
Előnye továbbá, hogy rendkívül egyszerűvé teszi a frontend és a backend közötti kódmegosztást, csökkentve ezzel a duplikációkat és erősítve a type safety-t.\cite{NextJS}

%----------------------------------------------------------------------------
\section{Adatbázis}
%----------------------------------------------------------------------------
Az adatbázis architektúra kiválasztása esetén két fő irányvonal volt a meghatározó: a hagyományos relációs adatbázis (pl. PostgreSQL, MySQL) és a NoSQL megoldások (pl. MongoDB, Google Firestore, AWS DynamoDB).

A technológia kiválasztása során fontos szempont volt az ACID elvek követése, a relációk egyszerű kezelése és a minél nagyobb típusbiztosság elérése a backend kódjában.

Ezeknek a szempontoknak a hagyományos relációs adatbázisok felelnek meg, ezen belül a PostgreSQL-re esett a választásom, mivel ez egy általam már ismert, sok funkciót támogató, ingyenes és nyílt forráskódú megoldás. \cite{Postgresql}

%----------------------------------------------------------------------------
\subsection{Az adatbázis elérése}
%----------------------------------------------------------------------------
Miután a backend és adatbázis technológiák és könyvtárak kiválasztásra kerültek, szükséges a kettő közötti kommunikációt biztosító könyvtár meghatározására.
NodeJS környezetben rendkívül nagy választék áll rendelkezésre, ezeket két nagy csoportra lehet bontani: ORM és query builder.

Míg az előbbi megoldás egy erős absztrakciós réteget képez az adatbázis szerkezete és a programkód között, addig a query builder egy, az SQL-hez közelebbi interfészt kínál a felhasználónak.
Ez utóbbi előnye, hogy a felhasználó által írt kód közelebb van a végső soron lefutó SQL-hez, így átláthatóbbak az egyes lekérések végeredményei.

A query builder megoldások közül is kiemelkedik a Prisma, amely egy NodeJS környezetben működő adatbázis kliens, ami a hangsúlyt a type safety-re helyezi.
Az adatbázis sémát egy speciális \lstinline|.prisma| kiterjesztésű fájlban tudjuk megírni, majd ezt a Prisma migrációs eszközével tudjuk az adatbázisunkba átvezetni.
Ezután lehetőségünk van a séma alapján a sémából TypeScript típusokat generálni, melyek használata nagyban megkönnyíti és felgyorsítja a fejlesztés menetét.

\begin{lstlisting}[language=Java, caption=Adatbázis beillesztés TypeORM környezetben]
const user = new User()
user.firstName = "Bob"
user.lastName = "Smith"
user.age = 25
await repository.save(user)
\end{lstlisting}

\begin{lstlisting}[language=Java, caption=Adatbázis beillesztés Prisma segítségével]
const user = await prisma.user.create({
  data: {
    firstName: "Bob",
    lastName: "Smith",
    age: 25
  }
})
\end{lstlisting}

A fentiek alapján a Prisma könyvtárat használtam az adatbázissal történő kommunikációra. \cite{Prisma}

\section{REST és GraphQL}

A fentieken túl utolsó lépésként szükséges volt eldönteni a frontend és a backend közötti kommunikáció módját.

A REST architektúra mellett ugyanis a közelmúltban megjelent a GraphQl. Ez egy, a REST API-nál flexibilisebb megoldást
kínál az adatok elérésére. Hátránya azonban, hogy sok extra kód és konfiguráció szükséges a használatához.
Ezzel szemben -- Next.js API routes használatával -- egy REST API végpont létrehozásához csupán a kiválasztott névvel kell egy fájlt
helyezni az \lstinline|src/pages/api| könyvtárba.

A fenti szempontokat figyelembe véve végül a hagyományos REST architektúra alkalmazása mellett döntöttem. \cite{REST}

\subsection{Kommunikáció megvalósítása}

A HTTP kérések küldéséhez rengeteg megoldás közül tudunk válogatni.
A legegyszerűbb a böngészőbe beépített fetch API\cite{fetchAPI}, ami egy egyszerű interfészt biztosít, azonban funkciókat tekintve erősen korlátozott.

A választásom végül a Vercel által fejlesztett SWR-re\cite{SWR} esett. Ez egy React könyvtár, ami a stale-while-revalidate
stratégiát alkalmazza. Segítségével könnyedén tudjuk a változtatásokat gyorsan és reaktívan követni a frontenden, miközben fejlett cache
funkcionalitást is nyújt.

\section{TypeScript}
A frontend illetve a Node.js világában lehetőségünk van arra, hogy a JavaScript helyett egy arra lefordítható
egyéb nyelvet használjunk fejlesztés közben. Az egyik ilyen lehetőség a TypeScript\cite{TypeScript}, amely egy már több éve jelenlévő,
gyorsan fejlődő alternatíva többek között olyan funkciókkal, mint a static type check, type inference és magas szintű IDE
integráció.

Az általam választott technológiák közül a Prisma kifejezetten épít a TypeScript nyújtotta előnyökre, illetve a Next.js
integrációhoz elegendő egy konfigurációs fájlt létrehozni, így evidens volt, hogy az alkalmazást TypeScript segítségével írjam meg.
