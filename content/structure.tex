\chapter{Az alkalmazás felépítése}

Az alkalmazás három fő rétege az adatbázis, a backend és a frontend. Az alábbi fejezet ezen három réteg architektúrális
felépítésésről valamint az egyes rétegek összeköttetéséről foglalkozik.

\begin{figure}[!ht]
  \centering
  \includegraphics[width=150mm, keepaspectratio]{figures/architecture-diagram.png}
  \caption{Az alkalmazás high-level architektúrája}
  \label{fig:Architecture}
\end{figure}

\section{Adatbázisséma}

\subsection{Tervezés}

Az adatbázisséma tervezéséhez a dbdiagram.io nevű platformfüggetlen, webes ER diagram tervező szoftvert használtam.
Ez egy saját fejlesztésű, DBML nevű DSL nyelvet használ a séma leírására, és lehetővé teszi ennek exportálását különféle formátumokba.

\begin{figure}[!ht]
\centering
\includegraphics[width=150mm, keepaspectratio]{figures/dbschema.png}
\caption{Az adatbázisséma ER diagramja.}
\label{fig:DBSchema}
\end{figure}

A séma tervezése során a Prisma által használt elnevezési konvenciókat használtam megkönnyítve a két technológia közötti átjárhatóságot.

\subsection{Implementáció}

A séma adatbázisba történő átvezetésére két lehetőségünk van. Az egyik, hogy a dbdiagram oldalról lehetőségünk van .sql kiterjesztésű fájlt letölteni,
ezt a létrehozott adatbázisunkon futtatni, majd a Prisma introspect funkcióját használva legenerálni hozzá a Prisma schema fájlt a backendünk számára.
A másik, egyszerűbb megoldás a Prisma migrate\footnote{A dolgozat írása idejében ez a funkció még experimental státuszban volt, de a használata során nem ütköztem problémákba.} használata.
Ez esetben nekünk manuálisan kell létrehozni a Prisma schema fájlt a korábbi diagram alapján, majd a
\begin{lstlisting}
prisma migrate save --experimental
prisma migrate up --experimental
\end{lstlisting}

parancsokat kiadva létrehozzuk és futtatjuk a Prisma migrációt.
Ez utóbbi megoldás előnyei, hogy egy központi helyen tudjuk kezelni a séma változásait, valamint ezt adatbázis-agnosztikus módon
tehetjük meg.

\begin{lstlisting}[caption=Séma leírása a Prisma DSL-ben]
generator client {
  provider = "prisma-client-js"
}

datasource db {
  provider = "postgresql"
  url      = env("DATABASE_URL")
}

model Book {
  id          Int           @id @default(autoincrement())
  title       String
  author      String?
  isbn        String?
  publisher   String?
  publishedAt Int?
  stockCount  Int?          @default(1)
  count       Int?          @default(1)
  notes       String?
  image       String?
  createdAt   DateTime?     @default(now())
  updatedAt   DateTime?     @default(now())
  orders      BookToOrder[]
  categories  Category[]
}

model BookToOrder {
  id       Int   @id @default(autoincrement())
  orderId  Int
  bookId   Int
  quantity Int?  @default(1)
  books    Book  @relation(fields: [bookId], references: [id])
  orders   Order @relation(fields: [orderId], references: [id])

  @@unique([bookId, orderId], name: "BookToOrder_book_order_unique")
}

model Category {
  id    Int    @id @default(autoincrement())
  name  String
  books Book[]
}

model Comment {
  id        Int       @id @default(autoincrement())
  text      String?
  createdAt DateTime? @default(now())
  userId    Int
  orderId   Int
  order     Order     @relation(fields: [orderId], references: [id])
  user      User      @relation(fields: [userId], references: [id])
}

model Order {
  id         Int           @id @default(autoincrement())
  userId     Int
  returnDate DateTime?
  status     orderstatus?  @default(PENDING)
  createdAt  DateTime?     @default(now())
  updatedAt  DateTime?     @default(now())
  user       User          @relation(fields: [userId], references: [id])
  books      BookToOrder[]
  comments   Comment[]
}

model User {
  id        Int       @id @default(autoincrement())
  email     String    @unique
  password  String
  name      String?
  createdAt DateTime? @default(now())
  role      userrole? @default(BASIC)
  comments  Comment[]
  orders    Order[]
}

enum orderstatus {
  PENDING
  RENTED
  RETURNED
  LATE
}

enum userrole {
  BASIC
  ADMIN
  EDITOR
}
\end{lstlisting}

A Prisma DSL-ben a dbdiagram.io-hoz hasonló módon tudjuk felvenni a relációkat. Ennek nagy előnye, hogy a táblák közötti kapcsolatokat
egyszerűen tudjuk modellezni, amit a \lstinline|prisma migrate| át tud vezetni az adatbázisunkba.

Ahogy a fenti kódrészletben is látszik, a Category és a Book modellek közötti kapcsolatot biztosító kapcsolótábla nem jelenik
meg a Prisma schema fájlban. Ez egy úgynevezett implicit kapcsolat, melyet a Prisma motorja automatikusan kezel és a migráció során
létrehozza a megfelelő táblát a két entitás között.

\subsection{Kapcsolódás az adatbázishoz}
A Prisma esetében az adatbáziskapcsolatot egy connection string segítségével tudjuk megadni, amit alapesetben
a megadott környezeti változóból (\lstinline|DATABASE_URL|) olvas ki. Ennek nagy előnye, hogy könnyen tudunk lokális
és remote adatbázisok között váltani, illetve egyszerűvé teszi a production-ben lévő kód és az adatbázis közötti
kapcsolat létrehozását.


\section{A backend felépítése}

\subsection{Next.js API routes}

A Next.js keretrendszer a 9-es verzió óta lehetővé teszi szerveroldali kód írását az alkalmazásunkhoz.
Ennek segítségével a \lstinline|pages/api| mappába helyetett fájljaink szolgálnak backendként. Minden ide helyezett fájl
egyben a nevének megfelelő API végpont lesz, tehát például a \lstinline|pages/api/books.ts|-ben lévő kódot a pages/api/books URL-en keresztül tudjuk elérni.

Támogatja továbbá a backend-oldali dinamikus routing-ot, azaz például a \lstinline|pages/api/books/[id].tsx| fájl a \lstinline|pages/api/<id>| URL-nek felel meg, ahol
az \lstinline|id| változót alábbi módon érhetjük el:

\begin{lstlisting}[language=Java, caption=Next.js dinamikus routing]
export default function handler(req, res) {
  const {
    query: { id },
  } = req

  res.end(`Book: ${id}`)
}
\end{lstlisting}

\subsection{next-connect}
Alapesetben a Next.js csak egy egyszerű interface-t biztosít nekünk, amin keresztül elérhetjük a HTTP kérés request és response objektumokat annak kezeléséhez.
Ez azonban nezhézkessé teszi a különböző kérések feldolgozását (pl. GET, POST és PUT), valamint különbőző kódrészletek egyszerű újrafelhasználását.

Ennek kényelmesebbé tételére döntöttem a next-connect könyvtár használata mellett, amellyel a fenti igények könnyedén megvalósíthatóak.
Az alábbi két kódrészletben szeretném bemutatni a főbb különbségeket.

\begin{lstlisting}[language=Java, caption=Default Next.js API routes]
export default function handler(req: NextApiRequest, res: NextApiResponse) {
  if (req.method === 'GET') {
    res.statusCode = 200
    res.setHeader('Content-Type', 'application/json')
    res.end(JSON.stringify({ name: 'John Doe' }))
  } else if (req.method === 'POST') {
    // Process a POST request
  }
}
\end{lstlisting}

\begin{lstlisting}[language=Java, caption=Kérés kezelése next-connect segítségével]
import nextConnect from "next-connect"

const handler = nextConnect<NextApiRequest, NextApiResponse>()

handler
  .get((req, res) => {
    res.json({ name: 'John Doe' })
  })
  .post((req, res) => {
    // Process POST request
  })

export default handler
\end{lstlisting}

\subsubsection{Middleware támogatás}
A Next.js alapesetben nem rendelkezik beépített middleware támogatással, emiatt bizonyos kódrészletek újrahasználása körülményes lehet.
A next-connect azonban ezt a folyamatot rendkívül egyszerűvé teszi, így könnyen lehet védett útvonalakat létrehozni például csak bejelentkezett
felhasználók számára.

\begin{lstlisting}[language=Java, caption=Middleware kezelés next-connect segítségével]
import nextConnect from "next-connect"
import requireLogin from "middleware/requireLogin"
import requireAdmin from "middleware/requireAdmin"
const handler = nextConnect<NextApiRequest, NextApiResponse>()

handler
  .get((req, res) => {
    res.json({ name: 'John Doe' })
  })
  .use(requreLogin)
  .post((req, res) => {
    // Process POST request
  })
  .use(requireAdmin)
  // Process other requests

export default handler
\end{lstlisting}

A fenti kódrészletben a POST kérés csak bejelentkezett felhasználók számára elérhető. Ezek egymás után is fűzhetőek, így komplex
igények is rendkívül egyszerűen megvalósíthatóak.


\section{A frontend felépítése}

Lorem ipsum

\section{Kódmegosztás}
Mivel a backend és frontend kódja egyazon git repository-ban van elhelyezve, a kettő között kódmegosztás szinte triviális.

Bármely, az \lstinline|src| mappán belül elhelyezett fájl felhasználható a szerver- és kliensoldali kód számára. Ennek megfelelően
az \lstinline|src/lib/interfaces.ts| és \lstinline|src/lib/constants.ts| fájlokban találhatóak a közösen használt interfészek és konstans változók.

Az általam használt Prisma ORM egy nagy előnye továbbá, hogy az általa generált interfészek és típusdefiníciók is használhatóak kliens- és szerveroldalon
egyaránt, ezáltal csökken a bugok mennyisége, és egy esetleges sémaváltozás esetén jóval könnyebb lekövetni az okozott változásokat.


\section{Képfeltöltés}

A könyvekhez lehetősév van borítóképet feltölteni, ezeket azonban valahol tárolni is kell.

Ennek a tárolására az Amazon S3 szolgáltatását választottam. A csatolt képet a frontend először elküldi a backendnek,
majd az az \lstinline|aws-sdk| könyvtárat használva feltölti a képet az S3 bucket-be, az adatbázisba csak a képet
azonosító generált kerül be. Így lehetőségünk van a képek egyszerű és adatbázisfüggetlen kezelésére.

\begin{figure}[!ht]
  \centering
  \includegraphics[width=125mm, keepaspectratio]{figures/s3-dashboard.png}
  \caption{Amazon S3 konzol}
  \label{fig:S3Console}
\end{figure}
