%----------------------------------------------------------------------------
\chapter{\bevezetes}
%----------------------------------------------------------------------------

A bevezető tartalmazza a diplomaterv-kiírás elemzését, történelmi előzményeit, a feladat indokoltságát (a motiváció leírását), az eddigi megoldásokat, és ennek tükrében a hallgató megoldásának összefoglalását.

A bevezető szokás szerint a diplomaterv felépítésével záródik, azaz annak rövid leírásával, hogy melyik fejezet mivel foglalkozik.

A Simonyi Károly Szakkollégiumban jelenleg üzemel egy könyvtár, ahonnan a hallgatók különféle tankönyveket és szórakoztató irodalmat van lehetőségük kölcsönözni.
Ennek az adminisztrációja egy megosztott Google Docs táblázaton keresztül történik, ami azonban a publikum felé nem nyilvános, a hallgatók a könyvtárban lévő könyvek elérhetőségét csak személyesen,
vagy az üzemeltetőknek írt emailen keresztül tudják ellenőrizni. Ez nagyban hátráltatja, hogy a hallgatókban tudatosuljon a könyvtár létezése, valamint  megnehezíti annak karbantartását és a kölcsönzés menetét is.

Ezen helyzet adta a motivációt arra, hogy készítsek egy, a hallgatók által könnyen hozzáférhető, és az üzemeltetők által kényelmesen menedzselhető webes alkalmazást a szakkollégium számára.
A cél egy reszponzív, web alapú felület tervezése és elkészítése a könyvtár könnyű használhatósága érdekében. (TODO: ezt talán szebben fogalmazni)

A dolgozat következő fejezeteiben bemutatom az alkalmazás tervezési és fejlesztési lépéseit.
Ezen belül szeretnék kitérni az általam végül kiválasztott technológiák bemutatására, a kész szoftver struktúrális felépítésére,
annak működési elvére, valamint az egyes komponensek közötti kommunikáció megvalósítására.

Végül kitérek a fejlesztést segítő eszközök bemutatására, majd bemutatom a jövőbeli fejlesztési lehetőségeket.
