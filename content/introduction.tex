%----------------------------------------------------------------------------
\chapter{\bevezetes}
%----------------------------------------------------------------------------

A bevezető tartalmazza a diplomaterv-kiírás elemzését, történelmi előzményeit, a feladat indokoltságát (a motiváció leírását), az eddigi megoldásokat, és ennek tükrében a hallgató megoldásának összefoglalását.

A bevezető szokás szerint a diplomaterv felépítésével záródik, azaz annak rövid leírásával, hogy melyik fejezet mivel foglalkozik.

- létező könyvtár a simonyiban, meglévő megoldások: excel táblázat az adminisztrációra, email ha valaki kölcsönözne, manuális és szétszórt egyeztetési folyamat,
amire más nem lát rá, körülményes adminisztráció a táblázatban, nem hozzáférhető a hallgatók számára

- megoldás összefoglalása: egy könnyen kezelhető webes felület, ami a könyvtárt használó hallgatónak és az azt adminisztráló szakkollégistának is egy egyszerű,
könnyen hozzáférhető és bővíthető platformot biztosít

fejezetek:
- tech stack kiválasztása
  - frontend, backend, db, rest/graphql
- alkalmazás felépítése
- hosting, jövőbeli deploy (kubernetes, docker)
- jövőbeli bővítési lehetőségek