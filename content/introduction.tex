%----------------------------------------------------------------------------
\chapter{\bevezetes}
%----------------------------------------------------------------------------

A bevezető tartalmazza a diplomaterv-kiírás elemzését, történelmi előzményeit, a feladat indokoltságát (a motiváció leírását), az eddigi megoldásokat, és ennek tükrében a hallgató megoldásának összefoglalását.

A bevezető szokás szerint a diplomaterv felépítésével záródik, azaz annak rövid leírásával, hogy melyik fejezet mivel foglalkozik.

A Simonyi Károly Szakkollégiumban jelenleg üzemel egy könyvtár, ahonnan a hallgatók különféle tankönyveket és szórakoztató irodalmat van lehetőségük kölcsönözni.
Ennek az adminisztrációja egy megosztott Google Docs táblázaton keresztül történik, ami azonban a publikum felé nem nyilvános, a hallgatók a könyvtárban lévő könyvek elérhetőségét csak személyesen,
vagy az üzemeltetőknek írt emailen keresztül tudják ellenőrizni. Ez nagyban hátráltatja, hogy a hallgatókban tudatosuljon a könyvtár létezése, valamint  megnehezíti annak karbantartását és a kölcsönzés menetét is.

Ezen helyzet adta a motivációt arra, hogy készítsek egy, a hallgatók által könnyen hozzáférhető, és az üzemeltetők által kényelmesen menedzselhető webes alkalmazást a szakkollégium számára.
A cél egy reszponzív, web alapú felület tervezése és elkészítése a könyvtár könnyű használhatósága érdekében. (TODO: ezt talán szebben fogalmazni)

TODO: szebben
fejezetek:
- Használt technológiák kiválasztása
  - frontend, backend, db, rest/graphql
- alkalmazás felépítése
  - adatbázis
    - db séma (dbdiagram export, prisma schema, prisma migrate)
    - prisma (schema file, migrations)
  - backend
    - nextjs file-based routing, middleware-ek
  - frontend
    - components/ mappa, file based routing
  - kódmegosztás
    - lib/ mappa, prisma típusok frontenden
- alkalmazás működése
  - authentikáció, authorizáció
  - adatlekérés a backendtől: useSWR
  - prisma
  - kosár
  - order leadása, menedzselése
  - keresés: postgres full text search, kihívások, megoldások
  - fájlfeltöltés: amazon S3

- tesztek (???)
- CI
  - github actions, eslint
- hosting, jövőbeli deploy (kubernetes, docker)
- jövőbeli bővítési lehetőségek
  - real-time chat websocket segítségével
  - hosted / dockerben futó fuzzy search (elastic, algolia)
  - authsch integráció a user-pass mellé
