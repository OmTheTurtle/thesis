%----------------------------------------------------------------------------
\chapter{\bevezetes}
%----------------------------------------------------------------------------

A Simonyi Károly Szakkollégiumban jelenleg üzemel egy könyvtár, ahonnan a hallgatók különféle tankönyveket és szórakoztató irodalmat van lehetőségük kölcsönözni.

A jelenlegi megoldás azonban nehézkes és nehezen fenntartható. Az adminisztráció egy megosztott Google Docs táblázaton keresztül történik. Itt ömlesztve találhatóak
a könyvek, azoknak a darabszáma és aktuális állapota (kölcsönözhető, kiadva). Ez a táblázat továbbá nem nyilvános, tehát a végfelhasználók nem tudják ellenőrizni egy
adott könyv elérhetőségét, illetve foglalást is csak email-en keresztül tudnak leadni.

Ez nagyban hátráltatja mind a könyvtár hirdetését a hallgatók számára, valamint lassúvá és bonyolulttá teszi a könyvtár menedzselését.

A fentebbi szituáció ösztönzött arra, hogy a jelenleginél optimálisabb megoldást keressek a könyvtár üzemeltetésére.

Önálló laboratóriumom és szakkollégiumi tagságom során megismerkedtem több webes technológiával, különös figyelmet szentelve a manapság népszerű
frontend technológiákra.

Ennek hatására kifejezetten frondend-fókuszú megoldást szerettem volna készíteni a minél jobb felhasználói élmény érdekében. Ennek megfelelően
a választásom a Next.js keretrendszerre és a serverless backend megoldásra esett. Mivel a kezdetektől TypeScript nyelvvel ismerkedtem meg és ezt használtam
evidens volt, hogy a szakdolgozatomban is a TypeScript-et és az azt támogató megoldásokat részesítem előnyben. Emiatt a backend és az adatbázis kapcsolatát
biztosító könyvtárak közül a Prisma került ki nyertesnek.

A dolgozatom a továbbiakban ennek a szoftvernek a megtervezéséről és működésének bemutatásáról fog szólni.

A második fejezetben a feladat által megkövetelt funkcionalitásokat. Ezek között szerepel a felhasználók kezelése, kölcsönzés leadása és
állapotának követése, valamint a könyvek, kategóriák és kölcsönzések adminisztrátor oldalról történő menedzselése.

A harmadik fejezetben bemutatom az általam az alkalmazás egyes rétegeihez használt technológiák, valamint az egyes rétegeket
összekötő könyvtárak kiválasztási folyamatait és a végső soron kiválasztott technológiákat.

A negyedik fejezet a weboldal magasszintű architektúrális felépítéséről értekezik. Bemutatom az egyes rétegek felépítését, illetve hogy
az előző fejezetben kiválasztott könyvtárak hogyan működnek együtt az alkalmazás különböző részeivel.

Az ötödik fejezet az alkalmazás által megvalósított funkciókat és az ehhez szükséges lépéseket és használt technikákat tartalmazza.

A hatodik fejezetben az alkalmazás tesztelése, a kiválasztott keretrendszer bemutatása és a tesztek felépítése szerepel.

Ezek után röviden összefoglalom az elkészült alkalmazást és kitérek a további fejlesztési lehetőségekre.
