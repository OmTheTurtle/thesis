\chapter{CI/CD}

A Continuous Integration / Continuous Delivery egy manapság már általánosan használt módszer az alkalmazások folyamatos tesztelésére és publikálására.

A Simonyi Könyvtár esetében igyekeztem ezeket a megközelítéseket alkalmazni a fejlesztés és deploy esetében is.

\section{Continuous Integration}

A forráskód tárolására a GitHub-ot használtam, így CI megoldásnak evidens volt a GitHub Actions használata.

Ez esetben elegendő egy \lstinline|.yaml| fájlt elhelyezni a \lstinline|.github/workflows/| mappába elhelyezni, és egy \lstinline|git push|
parancs kiadása után automatikusan lefut a szkriptünk. Ennek az állapotát a GitHub repository webes felületén tudjuk követni.

\section{Continuous Delivery}

Az alkalmazás hostolására két szolgáltatást használtam.

A frontend és backend közös deploymentjéhez a Vercel-t választottam. Ez rendkívül egyszerűen integrálható a Next.js keretrendszerrel,
elegendő a GitHub repository-val összekötni és bármiféle extra konfiguráció nélkül elérhető lesz az alkalmazásunk egy \lstinline|git push| parancs kiadása után.

A Vercel felületén lehetőségünk van a deployment státuszát ellenőrizni, korábbi deploymentre visszaállni, illetve a Next.js 10-es verziójától kezdve
már különböző analitikák monitorozására is.

Az adatbázishoz a Heroku ingyenes PostgreSQL szolgáltatását vettem igénybe. Ebben az esetben elegendő az adatbázishoz kapott
connection string-et a Vercel felületén a környezeti változók között megadni, és az alkalmazásunk hozzáfér az adatbázisunkhoz.
