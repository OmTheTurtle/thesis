\chapter{Összefoglalás}

Az alkalmazás elkészítése során sok eddig ismeretlen problémát kellett megoldanom, valamint megismertem
általam eddig kevésbé használt technológiákat és módszereket. A dolgozat elkészítése sok tapasztalattal látott
el, amiket kamatoztathatok a későbbiekben.

\section{Továbbfejlesztési lehetőségek}

\subsection{WebSocket}

A végfelhasználók és az oldal adminjai közötti kommunikáció jelenleg egyszerű HTTP kérésekkel történik,
ami bár az SWR könyvtárnak köszönhetően képes az oldal újratöltése nélkül frissíteni az üzeneteket tab refocus esetén,
de nem nyújt valós idejű kommunikációs élményt.

Ennek egyik fejlesztési lehetősége a WebSocket használata, mely segítségévez l a szerver is tud direkt
üzenetet küldeni a frontend felé, ezzel valódi real-time kommunikációt lehetővé téve.

\subsection{Értesítés rendszer}

A felhasználói élmény javításának egyik módja lehet egy értesítés rendszer implementálása. Ennek segítségével
az oldalon regisztráltak elsőkézből értesülnének, ha a foglalásuk állapota módosult vagy ahhoz új komment érkezett.

Ez megvalósítható csak a weboldalba épített módon, vagy email küldésével is, vagy akár a kettő kombinálásával.

\subsection{Keresés továbbfejlesztése}

A jelenleg megvalósított megoldás bár kulcsszavas keresésre jól alkalmazható, de elgépeléseket, illetve
szavak részleteire keresést nem támogat.

Egy fejlettebb, robosztusabb keresést biztosító rendszert kínál például az Algolia\footnote{https://www.algolia.com/} vagy az Elastic Search\footnote{https://www.elastic.co/} szolgáltatás,
melyek működhetnek cloud vagy self-hosted módban is.

\subsection{Auth.SCH integráció}

A kollégisták által igénybe vett weboldalaknál gyakran használt SSO megoldás az Auth.SCH\footnote{https://auth.sch.bme.hu/}, amivel
a hallgatók egy központosított azonosítóval tudnak bejelentkezni.

Ezt kihasználva a felhasználóknak nem kell külön jelszót megjegyezniük az alkalmazás használatához,
illetve az itt tárolt adatok alkalmazhatóak a felhasználó jogkörének megállapítására.
